\documentclass[12pt]{article}
\usepackage{enumitem}
\usepackage{amsmath}
\usepackage{amssymb}
\usepackage{amsfonts}
\usepackage{gensymb}
\usepackage{graphicx}
\usepackage{caption}
\usepackage{subcaption}
\usepackage{hyperref}
\usepackage{float}
\usepackage{braket}
\usepackage[margin=1.0in]{geometry}
\usepackage[style=chem-acs,backend=bibtex,articletitle=false]{biblatex}
\graphicspath{ {images/} }
\bibliography{mybib}

\begin{document}

\newcommand{\ind}{lm\epsilon}

\section{Theory}

\subsection{Projector Augmented Wave (PAW) Method}

\subsection{Projections}
\begin{equation}
\ket{\psi_{n\mathbf{k}}}=\ket{\widetilde{\psi}_{n\mathbf{k}}}+
\sum_{a,n,l,m}(\ket{\phi_{a\ind}}-\ket{\widetilde{\phi}_{a\ind}})
\braket{\widetilde{p}_{a\ind}|\widetilde{\psi}_{n\mathbf{k}}}
\label{eq:1}
\end{equation}

\subsubsection{Projection onto Basis from Different Structure}
In the pure physical picture of a quantum mechanical system, the eigenfunctions of the time
independent Hamiltonian form a complete basis for any function in real space. By expressing
a wavefunction from a system in the basis of the eigenstates of another system, one might
be able to learn about the nature of the wavefunction in question. For example, projecting
the wavefunction of a defect level in a periodic crystal onto the eigenfunctions of the
bulk crystal might yield information about how the defect level is constructed. For example,
the defect state might project primarily onto the conduction band, characterizing it
as a weakly perturbed host state. Such an understanding of defect levels could help select
corrections to perform on the defect system.\ref{eq:1}
\begin{equation}
\centering
\begin{split}
\braket{\psi_{Rn_1\mathbf{k}}|\psi_{Sn_2\mathbf{k}}}=\widetilde{O}+O_M+O_R+O_S+O_N\\
O_M=\sum_{a\in M_{RS}}\sum_{l,m}\sum_{\epsilon_1}\sum_{\epsilon_2}
\braket{\widetilde{\psi}_{Rn_1\mathbf{k}}|\widetilde{p}_{a\ind _1}}
(\braket{\phi_{a\ind _1}|{\phi_{a\ind _2}}}
-\bra{\widetilde{\phi}_{a\ind _1}}\ket{\widetilde{\phi}_{a\ind _2}})
\braket{\widetilde{p}_{a\ind _2}|\widetilde{\psi}_{Sn_2\mathbf{k}}}\\
O_R=\sum_{i\in N_R}\braket{\widetilde{\psi}_{Rn_1\mathbf{k}}|\widetilde{p}_i}
(\bra{\phi_i}-\bra{\widetilde{\phi}_i})
\ket{\widetilde{\psi}_{Sn_2\mathbf{k}}}\\
O_S=\sum_{j\in N_S}\bra{\widetilde{\psi}_{Rn_1\mathbf{k}}}(\ket{\phi_j}-\ket{\widetilde{\phi}_j})
\braket{\widetilde{p}_j|\widetilde{\psi}_{Sn_2\mathbf{k}}}\\
O_N=\sum_{i,j\in N_{RS}}\braket{\widetilde{\psi}_{Rn_1\mathbf{k}}|
\widetilde{p}_i}(\bra{\phi_i}-\bra{\widetilde{\phi}_i})
(\ket{\phi_j}-\ket{\widetilde{\phi}_j})\braket{\widetilde{p}_j|\widetilde{\psi}_{Sn_2\mathbf{k}}}
\end{split}
\label{eq:finp}
\end{equation}
Equation \ref{eq:finp} is derived in Appendix A.

\section{Implementation}

\section*{Appendix A: Derivation of Overlap Operators from PAW Formalism}
The starting point for this derivation is the transformation operator that
is performed on the pseudowavefunction to obtain the all electron wavefunction
\begin{equation}
T=1+\sum_a\sum_l\sum_m\sum_{\epsilon}(\ket{\phi_{a\ind}}-\ket{\widetilde{\phi}_{a\ind}})
\bra{\widetilde{p}_{a\ind}}=1+\sum_i(\ket{\phi_i}-\ket{\widetilde{\phi}_i})
\bra{\widetilde{p}_i}
\label{eq:T}
\end{equation}
where $\phi_{a\ind}$ are all electron (AE) partial waves, $\widetilde{\phi}_{a\ind}$
are pseudo (PS) partial waves, $\widetilde{p}_{a\ind}$ are projector functions,
$a$ are the site indices of each atom in the structure, and $l$, $m$, and $\epsilon$
specify a spherical harmonic and index which uniquely specify partial wave at a given
index. A summation over $i$ (or $j$, as below) represents a summation over $a$, $l$, $m$,
and $\epsilon$. For further details on the PAW method, including the physical significance
and construction of the partial waves and projector functions, see Blochl's original paper
and Kresse and Joubert's paper relating ultrasoft pseudopotentials and PAW.\cite{blochl}
The next step is to define and pseudo operator $\widetilde{A}$ for each operator $A$
such that $\bra{\psi}A\ket{\psi}=\bra{\widetilde{\psi}}\widetilde{A}\ket{\widetilde{\psi}}$.
Because $\ket{\psi}=T\ket{\widetilde{\psi}}$, one can write
\begin{equation}
\widetilde{A}=T^{\dagger}AT
\label{eq:op}
\end{equation}
One can then plug Equation \ref{eq:T} into Equation \ref{eq:op} to find
$$\widetilde{A}=[1+\sum_i\ket{\widetilde{p}_i}(\bra{\phi_i}-\bra{\widetilde{\phi}_i})]
A[1+\sum_j(\ket{\phi_j}-\ket{\widetilde{\phi}_j})
\bra{\widetilde{p}_j}]$$
\begin{equation}
\widetilde{A}=A+\sum_i\ket{\widetilde{p}_i}(\bra{\phi_i}-\bra{\widetilde{\phi}_i})A
+\sum_j A(\ket{\phi_j}-\ket{\widetilde{\phi}_j})\bra{\widetilde{p}_j}
+\sum_i\sum_j\ket{\widetilde{p}_i}(\bra{\phi_i}-\bra{\widetilde{\phi}_i})A
(\ket{\phi_j}-\ket{\widetilde{\phi}_j})\bra{\widetilde{p}_j}
\label{eq:res}
\end{equation}
Using the following relation,
$$(\bra{\phi_i}-\bra{\widetilde{\phi}_i})A(\ket{\phi_j}-\ket{\widetilde{\phi}_j})
=\bra{\phi_i}A\ket{\phi_j}-\bra{\widetilde{\phi}_i}A\ket{\phi_j}
-(\bra{\phi_i}-\bra{\widetilde{\phi}_i})A\ket{\widetilde{\phi}_j}$$
$$=\bra{\phi_i}A\ket{\phi_j}-\bra{\widetilde{\phi}_i}A(\ket{\phi_j}-\ket{\widetilde{\phi_j}})
-(\bra{\phi_i}-\bra{\widetilde{\phi}_i})A\ket{\widetilde{\phi}_j}
-\bra{\widetilde{\phi_i}}A\ket{\widetilde{\phi_j}}$$
one can rearrange Equation \ref{eq:res} in the manner of Blochl\cite{blochl}:
\begin{equation}
\begin{split}
\widetilde{A}=A+\sum_i\sum_j[\ket{\widetilde{p}_i}(\bra{\phi_i}A\ket{\phi_j}
-\bra{\widetilde{\phi_i}}A\ket{\widetilde{\phi_j}})\bra{\widetilde{p}_j}] \\
+\sum_i[(1-\sum_j\ket{\widetilde{p}_j}\bra{\widetilde{\phi}_j})A(\ket{\phi_i}-\ket{\widetilde{\phi_i}})
\bra{\widetilde{p}_i}+\ket{\widetilde{p}_i}
(\bra{\phi_i}-\bra{\widetilde{\phi}_i})A(1-\sum_j\ket{\widetilde{\phi}_j}\bra{\widetilde{p}_j})]
\end{split}
\label{eq:nonloc}
\end{equation}
When the operator $A$ is local, then $\sum_j\ket{\widetilde{\phi}_j}\bra{\widetilde{p}_j}=1$,
so the entire second line of Equation \ref{eq:nonloc} vanishes, giving Blochl's formulation:\cite{blochl}
\begin{equation}
\widetilde{A}=A+\sum_i\sum_j\ket{\widetilde{p}_i}(\bra{\phi_i}A\ket{\phi_j}
-\bra{\widetilde{\phi_i}}A\ket{\widetilde{\phi_j}})\bra{\widetilde{p}_j}
\label{eq:loc}
\end{equation}

However, while the overlap operator is local, the simplification in Equation \ref{eq:loc} assumes that
the summations over $i$ and $j$ go over the same species at the same locations in the same lattice;
only the third condition is always satisfied for systems of interest for projections between different
bases. Equation \ref{eq:res} is a more convenient representation than Equation \ref{eq:nonloc} for this
purpose because it is easily seen that all terms between projectors whose augmentation regions do not
overlap vanish. In addition, the POTCAR file in VASP only stores the PS and AE partial waves out to the
edge of the augmentation region, where they are equal but not nonzero, so it is useful to organize the terms
so that they are guaranteed to vanish outside the augmentation regions. Replacing $A$ with unity
allows the overlap between two bands
in different structures $R$ and $S$ with the same lattice to be specified (note that the
summation over $i$ is for structure $R$ and the summation over $j$ is for structure $S$):
\begin{equation}
\braket{\psi_{Rn_1\mathbf{k}}|\psi_{Sn_2\mathbf{k}}}=\widetilde{O}+O_1+O_2+O_3
\label{eq:ov1}
\end{equation}
$$\widetilde{O}=\braket{\widetilde{\psi}_{Rn_1\mathbf{k}}|\widetilde{\psi}_{Sn_2\mathbf{k}}}$$
$$O_1=\sum_i\braket{\widetilde{\psi}_{Rn_1\mathbf{k}}|\widetilde{p}_i}(\bra{\phi_i}-\bra{\widetilde{\phi}_i})
\ket{\widetilde{\psi}_{Sn_2\mathbf{k}}}$$
$$O_2=\sum_j\bra{\widetilde{\psi}_{Rn_1\mathbf{k}}}(\ket{\phi_j}-\ket{\widetilde{\phi}_j})
\braket{\widetilde{p}_j|\widetilde{\psi}_{Sn_2\mathbf{k}}}$$
$$O_3=\sum_i\sum_j\braket{\widetilde{\psi}_{Rn_1\mathbf{k}}|
\widetilde{p}_i}(\bra{\phi_i}-\bra{\widetilde{\phi}_i})
(\ket{\phi_j}-\ket{\widetilde{\phi}_j})\braket{\widetilde{p}_j|\widetilde{\psi}_{Sn_2\mathbf{k}}}$$
The pseudowavefunctions can be expanded as a summation of plane waves,
$$\widetilde{\psi}_{n\mathbf{k}}(\mathbf{r})=\sqrt{\frac{\omega_{\mathbf{k}}}{V}}
e^{i\mathbf{k}\cdot \mathbf{r}}\sum_{\mathbf{G}} C_{n\mathbf{k}\mathbf{G}}
e^{i\mathbf{G}\cdot \mathbf{r}}$$
so the overlap between two pseudowavefunctions per unit cell can be written as
$$\widetilde{O}=\braket{\widetilde{\psi}_{n_1\mathbf{k}_1}|\widetilde{\psi}_{n_2\mathbf{k}_2}}
=\delta_{\mathbf{k}_1,\mathbf{k}_2}
\omega_{\mathbf{k}} \sum_{\mathbf{G}} C^*_{n_1\mathbf{k}_1\mathbf{G}}C_{n_2\mathbf{k}_2\mathbf{G}}$$
In VASP, structures with the same energy, k-points, and lattice will have the same basis set,
so this projection is performed simply by reading plane wave coefficients from the VASP WAVECAR file.

It is important to simplify the calculation of the other terms in equation \ref{eq:ov1} as much
as possible because their calculation can be quite computationally intensive, and the number of
necessary calculations for projecting onto an entire basis set can scale with the number
of sites times the size of the basis set. One major simplification is that if a site $a$ in structure
$R$ and site $b$ in structure $S$ have the same species and position, $a$ and $b$ will only have
overlapping augmentation regions with each other and no other sites. Then, defining $O_{1a}$
as the summation over on-site terms for the identical sites $a$ and $b$ in $O_1$:
$$O_{1a}+O_{2a}+O_{3a}=\sum_{l,m}\sum_{\epsilon_1}\sum_{\epsilon_2}
\braket{\widetilde{\psi}_{Rn_1\mathbf{k}}|\widetilde{p}_{a\ind _1}}
(\braket{\phi_{a\ind _1}|{\phi_{a\ind _2}}}
-\bra{\widetilde{\phi}_{a\ind _1}}\ket{\widetilde{\phi}_{a\ind _2}})
\braket{\widetilde{p}_{a\ind _2}|\widetilde{\psi}_{Sn_2\mathbf{k}}}$$
which is the local operator solution derived by Blochl. All three terms must
be evaluated in full for the other sites, but terms in $O_3$ where $i$ and $j$ correspond
to sites with nonoverlapping augmentation spheres vanish. Therefore, if $M_{RS}$ is the set
of identical sites in the structures $R$ and $S$, $N_R$ and $N_S$ are the sets of sites
in $R$ and $S$ not in $M_{RS}$, and $N_{RS}$ is the set of \emph{pairs} of sites not in
$M$ with overlapping augmentation regions, then
$$
\braket{\psi_{Rn_1\mathbf{k}}|\psi_{Sn_2\mathbf{k}}}=\widetilde{O}+O_M+O_R+O_S+O_N
$$
$$
O_M=\sum_{a\in M_{RS}}\sum_{l,m}\sum_{\epsilon_1}\sum_{\epsilon_2}
\braket{\widetilde{\psi}_{Rn_1\mathbf{k}}|\widetilde{p}_{a\ind _1}}
(\braket{\phi_{a\ind _1}|{\phi_{a\ind _2}}}
-\bra{\widetilde{\phi}_{a\ind _1}}\ket{\widetilde{\phi}_{a\ind _2}})
\braket{\widetilde{p}_{a\ind _2}|\widetilde{\psi}_{Sn_2\mathbf{k}}}
$$
$$
O_R=\sum_{i\in N_R}\braket{\widetilde{\psi}_{Rn_1\mathbf{k}}|\widetilde{p}_i}
(\bra{\phi_i}-\bra{\widetilde{\phi}_i})
\ket{\widetilde{\psi}_{Sn_2\mathbf{k}}}
$$
$$
O_S=\sum_{j\in N_S}\bra{\widetilde{\psi}_{Rn_1\mathbf{k}}}(\ket{\phi_j}-\ket{\widetilde{\phi}_j})
\braket{\widetilde{p}_j|\widetilde{\psi}_{Sn_2\mathbf{k}}}
$$
$$
O_N=\sum_{i,j\in N_{RS}}\braket{\widetilde{\psi}_{Rn_1\mathbf{k}}|
\widetilde{p}_i}(\bra{\phi_i}-\bra{\widetilde{\phi}_i})
(\ket{\phi_j}-\ket{\widetilde{\phi}_j})\braket{\widetilde{p}_j|\widetilde{\psi}_{Sn_2\mathbf{k}}}
$$
which is Equation \ref{eq:finp}.

\section*{Appendix B: Comprehensive POTCAR Guide}

\printbibliography

\end{document}