\documentclass[12pt]{article}
\usepackage{authblk}
\usepackage{enumitem}
\usepackage{amsmath}
\usepackage{amssymb}
\usepackage{amsfonts}
\usepackage{gensymb}
\usepackage{graphicx}
\usepackage{caption}
\usepackage{subcaption}
\usepackage{hyperref}
\usepackage{float}
\usepackage{braket}
\usepackage[margin=1.0in]{geometry}
\usepackage[style=chem-acs,backend=bibtex,articletitle=false]{biblatex}
\graphicspath{ {../../Si_stuff/} }
\bibliography{mybib}

\begin{document}

\newcommand{\ind}{lm\epsilon}

\title{PAWpySeed: A parallelized Python/C package for analyzing
DFT wavefunctions in the PAW formalism}
\author[1]{Kyle Bystrom (kylebystrom@berkeley.edu)}
\affil[1]{UC Berkeley Department of Materials Science and Engineering}
\date{\today}
\maketitle
\abstract{This document serves as a technical reference for PAWpySeed.
The main novel development of PAWpySeed, covered in Section 1,
is the ability to evaluate the overlap operators of Kohn-Sham states
in the PAW formulism which belong to different structures. For examples,
one might want to project a defect level in a point defect structure
onto the bands of the bulk structure. Additional sections will
cover other developments as they are made.}

\section{Overlap Operators in PAW for Different Structures}

The equation of a PAW wavefunction is\cite{blochl}
\begin{equation}
\ket{\psi_{n\mathbf{k}}}=\ket{\widetilde{\psi}_{n\mathbf{k}}}+
\sum_{a,l,m,\epsilon}(\ket{\phi_{a\ind}}-\ket{\widetilde{\phi}_{a\ind}})
\braket{\widetilde{p}_{a\ind}|\widetilde{\psi}_{n\mathbf{k}}}
\label{eq:1}
\end{equation}
where the sum is over the sites $a$ in the structure and a set of
angular and radial quantum nubers $l$, $m$, and $\epsilon$ at each site.
The conventional overlap operator used for these wavefunctions is
\begin{equation}
\widetilde{A}=A+\sum_i\sum_j\ket{\widetilde{p}_i}(\bra{\phi_i}A\ket{\phi_j}
-\bra{\widetilde{\phi_i}}A\ket{\widetilde{\phi_j}})\bra{\widetilde{p}_j}
\label{eq:loc}
\end{equation}
where $i$ and $j$ are abbreviations for a particular index set
$a,l,m,\epsilon$. This overlap operator requires that the augmentation
regions of the two wavefunctions are same. Since these augmentation regions
are defined by the sites in the structure, the structures must be the same,
so a new formalism is necessary to calculate overlap operators for wavefunctions
in different structures.

This section procedes first by deriving the main formula for calculating
the overlap operator for KS states in different structures, and then
presents a runtime scaling table for each operation involved in evaluating
the formula. The algorithm for each operation is then discussed

\subsection{Derivation of Formula}

The following section derives an equation for the overlap operator between one
Kohn-Sham single particle state of one structure R and one Kohn-Sham single
particle state of another structure S, where R and S share a common lattice
and the DFT wavefunctions are constructed with the same plane-wave PAW basis set.
It should be noted this derivation might better be described as a manipulation
of terms of the PAW pseudo-operator, because its main purpose is to present
the overlap operator in a method convenient for computation rather than to derive
an entirely new expression. The derivation also makes particular note of how
the expression of the overlap operator relates to interfacing with the VASP code,
though the expressions are applicable to any PAW code.

The starting point for this derivation is the transformation operator that
is performed on the pseudowavefunction to obtain the all electron wavefunction
\begin{equation}
T=1+\sum_a\sum_l\sum_m\sum_{\epsilon}(\ket{\phi_{a\ind}}-\ket{\widetilde{\phi}_{a\ind}})
\bra{\widetilde{p}_{a\ind}}=1+\sum_i(\ket{\phi_i}-\ket{\widetilde{\phi}_i})
\bra{\widetilde{p}_i}
\label{eq:T}
\end{equation}
where $\phi_{a\ind}$ are all electron (AE) partial waves, $\widetilde{\phi}_{a\ind}$
are pseudo (PS) partial waves, $\widetilde{p}_{a\ind}$ are projector functions,
$a$ are the site indices of each atom in the structure, and $l$, $m$, and $\epsilon$
specify a spherical harmonic and index which uniquely specify partial wave at a given
index. A summation over $i$ (or $j$, as below) represents a summation over $a$, $l$, $m$,
and $\epsilon$. For further details on the PAW method, including the physical significance
and construction of the partial waves and projector functions, see Blochl's original paper
and Kresse and Joubert's paper relating ultrasoft pseudopotentials and PAW.\cite{blochl,vasp4}
The next step is to define a pseudo operator $\widetilde{A}$ for each operator $A$
such that $\bra{\psi}A\ket{\psi}=\bra{\widetilde{\psi}}\widetilde{A}\ket{\widetilde{\psi}}$.
Because $\ket{\psi}=T\ket{\widetilde{\psi}}$, one can write
\begin{equation}
\widetilde{A}=T^{\dagger}AT
\label{eq:op}
\end{equation}
One can then plug Equation \ref{eq:T} into Equation \ref{eq:op} to find
\begin{equation}
\begin{split}
\widetilde{A} = & [1+\sum_i\ket{\widetilde{p}_i}(\bra{\phi_i}-\bra{\widetilde{\phi}_i})]
A[1+\sum_j(\ket{\phi_j}-\ket{\widetilde{\phi}_j})
\bra{\widetilde{p}_j}]\\
\widetilde{A} = & A+\sum_i\ket{\widetilde{p}_i}(\bra{\phi_i}-\bra{\widetilde{\phi}_i})A
+\sum_j A(\ket{\phi_j}-\ket{\widetilde{\phi}_j})\bra{\widetilde{p}_j}\\
& +\sum_i\sum_j\ket{\widetilde{p}_i}(\bra{\phi_i}-\bra{\widetilde{\phi}_i})A
(\ket{\phi_j}-\ket{\widetilde{\phi}_j})\bra{\widetilde{p}_j}
\end{split}
\label{eq:res}
\end{equation}

Using the following relation,
\begin{equation}
\begin{split}
(\bra{\phi_i}-\bra{\widetilde{\phi}_i})A(\ket{\phi_j}-\ket{\widetilde{\phi}_j})
& =\bra{\phi_i}A\ket{\phi_j}-\bra{\widetilde{\phi}_i}A\ket{\phi_j}
-(\bra{\phi_i}-\bra{\widetilde{\phi}_i})A\ket{\widetilde{\phi}_j}\\
& =\bra{\phi_i}A\ket{\phi_j}-\bra{\widetilde{\phi}_i}A(\ket{\phi_j}-\ket{\widetilde{\phi_j}})
-(\bra{\phi_i}-\bra{\widetilde{\phi}_i})A\ket{\widetilde{\phi}_j}
-\bra{\widetilde{\phi_i}}A\ket{\widetilde{\phi_j}}
\end{split}
\end{equation}

one can rearrange Equation \ref{eq:res} in the manner of Blochl\cite{blochl}:
\begin{equation}
\begin{split}
\widetilde{A}= & A+\sum_i\sum_j[\ket{\widetilde{p}_i}(\bra{\phi_i}A\ket{\phi_j}
-\bra{\widetilde{\phi_i}}A\ket{\widetilde{\phi_j}})\bra{\widetilde{p}_j}] \\
& +\sum_i[(1-\sum_j\ket{\widetilde{p}_j}\bra{\widetilde{\phi}_j})A(\ket{\phi_i}-\ket{\widetilde{\phi_i}})
\bra{\widetilde{p}_i}+\ket{\widetilde{p}_i}
(\bra{\phi_i}-\bra{\widetilde{\phi}_i})A(1-\sum_j\ket{\widetilde{\phi}_j}\bra{\widetilde{p}_j})]
\end{split}
\label{eq:nonloc}
\end{equation}
When the operator $A$ is local, then $\sum_j\ket{\widetilde{\phi}_j}\bra{\widetilde{p}_j}=1$,
so the entire second line of Equation \ref{eq:nonloc} vanishes, giving Blochl's formulation:\cite{blochl}
\begin{equation}
\widetilde{A}=A+\sum_i\sum_j\ket{\widetilde{p}_i}(\bra{\phi_i}A\ket{\phi_j}
-\bra{\widetilde{\phi_i}}A\ket{\widetilde{\phi_j}})\bra{\widetilde{p}_j}
\label{eq:loc1}
\end{equation}

However, while the overlap operator is local, the simplification in Equation \ref{eq:loc1} assumes that
the summations over $i$ and $j$ go over the same species at the same locations in the same lattice;
only the third condition is always satisfied for systems of interest for projections between different
bases. Equation \ref{eq:res} is a more convenient representation than Equation \ref{eq:nonloc} for this
purpose because it is easily seen that all terms between projectors whose augmentation regions do not
overlap vanish. In addition, the POTCAR file in VASP only stores the PS and AE partial waves out to the
edge of the augmentation region, where they are equal but not nonzero, so it is useful to organize the terms
so that they are guaranteed to vanish outside the augmentation regions. Replacing $A$ with unity
allows the overlap between two bands
in different structures $R$ and $S$ with the same lattice to be specified (note that the
summation over $i$ is for structure $R$ and the summation over $j$ is for structure $S$):
\begin{equation}
\begin{split}
\braket{\psi_{Rn_1\mathbf{k}}|\psi_{Sn_2\mathbf{k}}} & =\widetilde{O}+O_1+O_2+O_3\\
\widetilde{O} & =\braket{\widetilde{\psi}_{Rn_1\mathbf{k}}|\widetilde{\psi}_{Sn_2\mathbf{k}}}\\
O_1 & =\sum_i\braket{\widetilde{\psi}_{Rn_1\mathbf{k}}|\widetilde{p}_i}(\bra{\phi_i}-\bra{\widetilde{\phi}_i})
\ket{\widetilde{\psi}_{Sn_2\mathbf{k}}}\\
O_2 & =\sum_j\bra{\widetilde{\psi}_{Rn_1\mathbf{k}}}(\ket{\phi_j}-\ket{\widetilde{\phi}_j})
\braket{\widetilde{p}_j|\widetilde{\psi}_{Sn_2\mathbf{k}}}\\
O_3 & =\sum_i\sum_j\braket{\widetilde{\psi}_{Rn_1\mathbf{k}}|
\widetilde{p}_i}(\bra{\phi_i}-\bra{\widetilde{\phi}_i})
(\ket{\phi_j}-\ket{\widetilde{\phi}_j})\braket{\widetilde{p}_j|\widetilde{\psi}_{Sn_2\mathbf{k}}}
\end{split}
\label{eq:ov1a}
\end{equation}

The pseudowavefunctions can be expanded as a summation of plane waves,
$$\widetilde{\psi}_{n\mathbf{k}}(\mathbf{r})=\sqrt{\frac{\omega_{\mathbf{k}}}{V}}
e^{i\mathbf{k}\cdot \mathbf{r}}\sum_{\mathbf{G}} C_{n\mathbf{k}\mathbf{G}}
e^{i\mathbf{G}\cdot \mathbf{r}}$$
so the overlap between two pseudowavefunctions per unit cell can be written as
$$\widetilde{O}=\braket{\widetilde{\psi}_{n_1\mathbf{k}_1}|\widetilde{\psi}_{n_2\mathbf{k}_2}}
=\delta_{\mathbf{k}_1,\mathbf{k}_2}
\omega_{\mathbf{k}} \sum_{\mathbf{G}} C^*_{n_1\mathbf{k}_1\mathbf{G}}C_{n_2\mathbf{k}_2\mathbf{G}}$$

In VASP, structures with the same energy, k-points, and lattice will have the same basis set,
so this projection is performed simply by reading plane wave coefficients from the VASP WAVECAR file.

It is important to simplify the calculation of the other terms in equation \ref{eq:ov1a} as much
as possible because their calculation can be quite computationally intensive, and the number of
necessary calculations for projecting onto an entire basis set can scale with the number
of sites times the size of the basis set. One major simplification is that if a site $a$ in structure
$R$ and site $b$ in structure $S$ have the same species and position, $a$ and $b$ will only have
overlapping augmentation regions with each other and no other sites. Then, defining $O_{1a}$
as the summation over on-site terms for the identical sites $a$ and $b$ in $O_1$:
$$O_{1a}+O_{2a}+O_{3a}=\sum_{l,m}\sum_{\epsilon_1}\sum_{\epsilon_2}
\braket{\widetilde{\psi}_{Rn_1\mathbf{k}}|\widetilde{p}_{a\ind _1}}
(\braket{\phi_{a\ind _1}|{\phi_{a\ind _2}}}
-\braket{\widetilde{\phi}_{a\ind _1}|\widetilde{\phi}_{a\ind _2}})
\braket{\widetilde{p}_{a\ind _2}|\widetilde{\psi}_{Sn_2\mathbf{k}}}$$
which is the local operator solution derived by Blochl. All three terms must
be evaluated in full for the other sites, but terms in $O_3$ where $i$ and $j$ correspond
to sites with nonoverlapping augmentation spheres vanish. Therefore, if $M_{RS}$ is the set
of identical sites in the structures $R$ and $S$, $N_R$ and $N_S$ are the sets of sites
in $R$ and $S$ not in $M_{RS}$, and $N_{RS}$ is the set of \emph{pairs} of sites not in
$M$ with overlapping augmentation regions, then
\begin{align}
\braket{\psi_{Rn_1\mathbf{k}}|\psi_{Sn_2\mathbf{k}}}&=\widetilde{O}+O_M+O_R+O_S+O_N\\
O_M&=\sum_{a\in M_{RS}}\sum_{l,m}\sum_{\epsilon_1}\sum_{\epsilon_2}
\braket{\widetilde{\psi}_{Rn_1\mathbf{k}}|\widetilde{p}_{a\ind _1}}
(\braket{\phi_{a\ind _1}|{\phi_{a\ind _2}}}
-\braket{\widetilde{\phi}_{a\ind _1}|\widetilde{\phi}_{a\ind _2}})
\braket{\widetilde{p}_{a\ind _2}|\widetilde{\psi}_{Sn_2\mathbf{k}}}\\
O_R&=\sum_{i\in N_R}\braket{\widetilde{\psi}_{Rn_1\mathbf{k}}|\widetilde{p}_i}
(\bra{\phi_i}-\bra{\widetilde{\phi}_i})
\ket{\widetilde{\psi}_{Sn_2\mathbf{k}}}\\
O_S&=\sum_{j\in N_S}\bra{\widetilde{\psi}_{Rn_1\mathbf{k}}}(\ket{\phi_j}
-\ket{\widetilde{\phi}_j})
\braket{\widetilde{p}_j|\widetilde{\psi}_{Sn_2\mathbf{k}}}\\
O_N&=\sum_{i,j\in N_{RS}}\braket{\widetilde{\psi}_{Rn_1\mathbf{k}}|
\widetilde{p}_i}(\bra{\phi_i}-\bra{\widetilde{\phi}_i})
(\ket{\phi_j}-\ket{\widetilde{\phi}_j})
\braket{\widetilde{p}_j|\widetilde{\psi}_{Sn_2\mathbf{k}}}
\label{eq:overlap}
\end{align}

\begin{table}
\centering
\caption{Runtime scaling functions for each component of the code and
definitions for shorthand symbols to express runtime. Approximate
scales with the number of electrons are also shown.
*The frequency refers to how often the routine is called. "Per band"
indicates that the routine runs once every time a band from one structure
is projected onto all the bands of a basis structure. "Per structure"
indicates that the routine is a "setup" routine that must be called once
for each structure and corresponding band structure to do perform projections. "Per lattice"
operations are required once per unique lattice used.
"Per structure pair" is the same as "per structure" except that the routine
is only run once for a pair of structures with the same lattice and plane-wave
basis set, the bands of one such structure to be used as a basis set for the other.
**Number of sites flexibly refers to the number of sites relevant to the calculation,
which worst-case scales with the total number of sites in the structure. For example,
calculating $O_M$ and $O_N$ only require the sites in sets $M$ and $N_RS$, respectively.}
\label{tab:runtime}

\begin{tabular}{|c|c|c|}
\hline
Computational Task & $\Theta$ & Frequency*\\
\hline
$\widetilde{O}$ & $BKSW \sim n^2$ & per band\\
$O_M$ & $BKSNP \sim n^2$ & per band\\
$O_R$ and $O_S$ & $BKSNP \sim n^2$ & per band\\
$O_N$ & $BKSNP \sim n^2$ & per band\\
$\braket{\widetilde{p}_i|\widetilde{\psi}_{n\mathbf{k}}}$ & $BKSF\mathrm{log}(F)
\sim n^2\mathrm{log}(n)$ & per structure\\
$(\bra{\phi_i}-\bra{\widetilde{\phi}_i})
\ket{\widetilde{\psi}_{n\mathbf{k}}}$ & $BKSNPW \sim n^3$ & per structure\\
spherical Bessel transform partial waves & $NPG\mathrm{log}(G) \sim n$ & per structure pair\\
project plane waves onto partial waves & $EPKW \sim n$ & per lattice\\
projections for overlapping aug spheres & $NPL \sim n$ & per structure pair\\
\hline
\end{tabular}

\begin{tabular}{c|c}
Symbol & Definition\\
\hline
B & number of bands\\
E & number of elements\\
F & size of FFT grid\\
G & size of partial wave radial grid\\
K & number of k-points\\
L & size of Legendre-Gauss grid around ions\\
N & number of sites**\\
P & number of projector functions\\
S & number of spin states\\
W & number of plane waves\\
n & number of electrons (approximate scaling)
\end{tabular}
\end{table}

\subsection{Overlap of Pseudowavefunction ($\widetilde{O}$)}

The pseudowavefunction is expanded as a sum of plane waves:

\begin{equation}
\widetilde{\psi}_{n\mathbf{k}} =\frac{1}{\sqrt{V}}\sum_{\mathbf{G}}
C_{n\mathbf{k}\mathbf{G}}e^{i\mathbf{k'}\cdot \mathbf{r}}
\label{eq:pswf}
\end{equation}

where $V$ is the volume of the unit cell, $\mathbf{k}$ is the k-point,
$n$ is the band index, $\mathbf{r}$ is the position, and $C_{n\mathbf{k}\mathbf{G}}$
are the plane-wave coefficients for the plane-waves with vector $\mathbf{k+G}$.

Since plane waves are orthogonal and normalized to $V$ when integrated over the
unit cell, the overlap of two pseudowavefunctions is evaluated easily
by summing over plane-wave coefficients as described in the derivation section
above.

\subsection{Concentric Augmentation Spheres ($O_M$}

Integrals of the type $\braket{\widetilde{\psi}_{Rn_1\mathbf{k}}|\widetilde{p}_{a\ind _1}}$
between a pseudowavefunction and projector function are evaluated using a real-space
FFT grid, as with VASP with LREAL=TRUE.\cite{vasp} Integrals of the
type $\braket{\phi_{a\ind _1}|{\phi_{a\ind _2}}}$ between partial waves
are evaluated by simple radial integration. This is possible because the augmentation
spheres are concentric, so the spherical harmonics for the partial waves
are orthonormal.

\subsection{Partial Waves Overlapping with Pseudowavefunction ($O_R$, $O_S$)}

These integrals are a little bit trickier because it is necessary to project
a smoothly varying pseudowavefunction onto a rapidly varying AE partial wave.
Performing such projections can become prohibitively computationally expensive
for large cells.
Taking advantage of the orthogonality of plane waves, this projection can be
done in real space. Since a plane wave can be expanded around an arbitrary origin
in space (to a phase factor) using Rayleigh expansion:

\begin{equation}
e^{i\mathbf{k} \cdot \mathbf{r}} = 4\pi \sum_{l=0}^{\infty}\sum_{m=-l}^{l}
i^l j_l(kr)Y_l^{m*}(\mathbf{\hat{k}})Y_l^m(\mathbf{\hat{r}})
\label{eq:pwexp}
\end{equation}

partial waves can be Fourier transformed into reciprocal space by evaluating
overlap integrals with spherical Bessel functions. This is done using the
$O(NlogN)$ NUMSBT algorithm developed by Talman.\cite{TALMAN} Then, all
frequency components greater than the FFT grid density can be set to 0
because those plane-waves are orthogonal to the finite plane-wave
basis set of the pseudowavefunctions,
and the partial waves can be transformed back into real space, also using
the NUMSBT algorithm. This results in a smooth partial waves
for which $(\bra{\phi_i}-\bra{\widetilde{\phi}_i})\ket{\widetilde{\psi}_{n\mathbf{k}}}$
can be evaluated in real space.

\subsection{Partial Wave Overlap on Non-Orthgonal Augmentation Spheres ($O_N$)}

The $O_N$ term appears similar to the $O_M$ term, except that the integrals
$(\bra{\phi_i}-\bra{\widetilde{\phi}_i})(\ket{\phi_j}-\ket{\widetilde{\phi}_j})$
are more difficult to evaluate because the partial waves are centered
at different sites. However, transforming these partial waves into
reciprocal space using the spherical Bessel transform allows the
overlap integrals to be evaluated using Equation (47) in Talman's
NUMSBT paper.\cite{TALMAN}.

\printbibliography

\end{document}
